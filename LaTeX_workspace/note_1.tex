\section{Motivation}
Lyapunov spectrum $\{\lambda_i\}$ and Lyapunov basis $\{\psi_i\}$ provide a natural discription of quantum circuits (QCs) with measurement.
Especially, the Lyapunov gap $\lambda_1 - \lambda_2$ can be used to characterise the relaxation to the steady state.
This picture reminds us the Liouvillian gap $\Delta$.
Also, the evolution of a state in QCs with measurement can be seen as single trajectory of a dissipative system and the Lindblad master equation (GKSL form) can be unravelling into an ensemble of such trajectories.
This picture prompts us to apply the Lyapunov analysis to GKSL form.
By analogy with the Liouvillian gap $\Delta$, we try to analysis the relation between $\{\lambda_i\}$ and 3 different physical problems and this gives us the outline of this work.
\begin{enumerate}
    \item the algorithm of Lyapunov analysis for GKSL form.
    \item compare QCs with measurement and GKSL form.
    \item the relation between $\{\lambda_i\}$ and dissipative phase transition (DPTs).
    \item the relation between $\{\lambda_i\}$ and the relaxation to the steady state.
    \item the relation between $\{\lambda_i\}$ and the topology. This is a analogy of topological origin of closure of $\Delta$.
\end{enumerate}




\begin{figure}[hbpt]
    \centering
    \includegraphics[width=1.0\linewidth]{pic/Lyap_convergence1.pdf}\\
    \includegraphics[width=1.0\linewidth]{pic/Lyap_convergence2.pdf}
\end{figure}

\begin{figure}[hbpt]
    \centering
    \includegraphics[width=1.0\linewidth]{pic/Lyap_dpt_trj.pdf}\\
    \includegraphics[width=1.0\linewidth]{pic/Lyap_dpt_mean.pdf}
\end{figure}